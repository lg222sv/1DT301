%----------------------------------------------------------------------------------------
%
% A LaTeX-template for 1DV510. Modified and translated by Björn Lindenberg at LNU.
% Based on an original master thesis template created by Marcus Wilhelmsson at LNU.
%
%----------------------------------------------------------------------------------------

% Settings and document configuration

\documentclass[a4paper,12pt]{article} 
\usepackage[T1]{fontenc} 
\usepackage{times} 
\usepackage[swedish,english]{babel} 
\usepackage[utf8]{inputenc} 
\usepackage{dtk-logos} 
\usepackage{wallpaper} 
\usepackage[absolute]{textpos} 
\usepackage[top=2cm, bottom=2.5cm, left=3cm, right=3cm]{geometry} 
\usepackage[parfill]{parskip} 
\usepackage{csquotes} 
\usepackage{float} 
\usepackage{lipsum} % Used for dummy text. Can be removed.
\usepackage{listings, color}
\usepackage{graphicx}

\lstdefinestyle{Asm}{
  belowcaptionskip=1\baselineskip,
  breaklines=true,
  frame=L,
  xleftmargin=\parindent,
  language=[x86masm]Assembler,
  showstringspaces=false,
  basicstyle=\footnotesize\ttfamily,
  keywordstyle=\bfseries\color{purple!40!black},
  commentstyle=\itshape\color{green!40!black},
  identifierstyle=\color{blue},
  stringstyle=\color{orange},
}

% Fontsizes for section headings.
\usepackage{sectsty} 
\sectionfont{\fontsize{14}{15}\selectfont}
\subsectionfont{\fontsize{12}{15}\selectfont}
\subsubsectionfont{\fontsize{12}{15}\selectfont}

%----------------------------------------------------------------------------------------
%	This part is used for the text box on the title page
%----------------------------------------------------------------------------------------
\newsavebox{\mybox}
\newlength{\mydepth}
\newlength{\myheight}

\newenvironment{sidebar}%
{\begin{lrbox}{\mybox}\begin{minipage}{\textwidth}}%
{\end{minipage}\end{lrbox}%
 \settodepth{\mydepth}{\usebox{\mybox}}%
 \settoheight{\myheight}{\usebox{\mybox}}%
 \addtolength{\myheight}{\mydepth}%
 \noindent\makebox[0pt]{\hspace{-20pt}\rule[-\mydepth]{1pt}{\myheight}}%
 \usebox{\mybox}}

%----------------------------------------------------------------------------------------
%	Title
%----------------------------------------------------------------------------------------
\newcommand\BackgroundPic{
    \put(-2,-3){
    \includegraphics[keepaspectratio,scale=0.3]{img/lnu_etch.png} % Background image
    }
}
\newcommand\BackgroundPicLogo{
    \put(30,740){
    \includegraphics[keepaspectratio,scale=0.10]{img/logo.png} % LNU logo
    }
}

\title{
\vspace{-8cm}
\begin{sidebar}
    \vspace{10cm}
    \normalfont \normalsize
    \huge Computer Technology I\\ % Main title
    \vspace{-1.3cm}
\end{sidebar}
\vspace{3cm}
\begin{flushleft}
    \huge Lab. 1 : How to use the PORTs, Digital input/output, Subroutine call % Subtitle
     \small \\ \emph{}
\end{flushleft}
\null
\vfill
\begin{textblock}{5}(10,13)
\begin{flushright}
\begin{minipage}{\textwidth}
\begin{flushleft} \large
\emph{Authors:}\textsc{ Leonardo PEDRO, Loïc GALLAND}\\  % Author
\emph{Supervisor:}  \textsc{} \\  % Author
\emph{Semester:} Autumn 2019\\ % Semester
\emph{Area:} Computer Science \\ % Area
\emph{Course code:} 1DT301 % Course
\end{flushleft}
\end{minipage}
\end{flushright}
\end{textblock}
}

\date{} % Empty date command. Use \today inside for today's date.
\author{} % Normally one would use this to define authors. However in this case the title command takes care of everything, so we leave the field empty to get rid of warnings. 

\begin{document}

\pagenumbering{gobble} % Turn off page numbering
\newgeometry{left=5cm}
\AddToShipoutPicture*{\BackgroundPic} % Adds the background image to the title page
\AddToShipoutPicture*{\BackgroundPicLogo} % Adds the logo to the title page
\maketitle % Prints the title
\restoregeometry
\clearpage

\pagenumbering{roman} % Roman page numbering for abstract page


\selectlanguage{english}

\newpage

\pagenumbering{gobble} % Turn off page numbering
\tableofcontents 

\newpage
\pagenumbering{arabic} % Turn on page numbering


\section{Task 1}

\textit{Write a program in Assembly language to light LED 2.
You can use any of the four ports, but start with PORTB.
The program should be very short! How many instructions is minimum number?}

\lstset{style=Asm}
\begin{lstlisting}
;>>>>>>>>>>>>>>>>>>>>>>>>>>>>>>>>>>>>>>>>>>>>>>>>>>>>>>>>>>>
; 1DT301, Computer Technology I
; Date: 2019-09-09
; Author:
; Loic GALLAND
; Leonardo PEDRO
;
; Lab number: 1
; Title: How to use the PORTs. Digital input/output. Subroutine call.
;
; Hardware: STK600, CPU ATmega2560
;
; Function: Program to light up the LED number 2

; Input ports: NO inputs ports in this Task
;
; Output ports: The portB is used as an output port
;
; Subroutines: If applicable.
; Included files: m2560def.inc
;
; Other information:
;
; Changes in program: (Description and date)
;
;<<<<<<<<<<<<<<<<<<<<<<<<<<<<<<<<<<<<<<<<<<<<<<<<<<<<<<<<<<<

.includes "m2560def.inc"
ldi r16, 0xFF
out DDRB, r16

ldi r16 , 0b11111011
out portB, r16
\end{lstlisting}
To be able to light up the LEDs we need 4 lines of code. The first line is to store into the register r16 the value 0xFF. In the second line the register r16 is loaded to DDRB (Data Direction Register of port B). In the third line the desired binary code is stored into register r16. The binary number will determine which LED will light up. In the last line the register r16 is loaded into the PortB (Data Register of Port B).
\newpage
This is the flowchart of the task 1:
\begin{center}
\includegraphics{img/Task1.png}
\end{center}

\newpage
\section{Task 2}
\textit{Write a program in Assembly language to read the switches and light the corresponding
LED.
Example: When you press SW5, LED5 so should light.
Make an initialization part of the program and after that an infinite loop.}
\lstset{style=Asm}
\begin{lstlisting}
;>>>>>>>>>>>>>>>>>>>>>>>>>>>>>>>>>>>>>>>>>>>>>>>>>>>>>>>>>>>
; 1DT301, Computer Technology I
; Date: 2019-09-09
; Author:
; Loic GALLAND
; Leonardo PEDRO
;
; Lab number: 1
; Title: How to use the PORTs. Digital input/output. Subroutine call.
;
; Hardware: STK600, CPU ATmega2560
;
; Function: Program to light up the LED correponding to the switch. EX: (Switch number 1 will light up LED number 1)

; Input ports: PortA is used as input to get the information from the switches 
;
; Output ports: The portB is used as an output ports to control the LEDs
;
; Subroutines: If applicable.
; Included files: m2560def.inc
;
; Other information:
;
; Changes in program: (Description and date)
;
;<<<<<<<<<<<<<<<<<<<<<<<<<<<<<<<<<<<<<<<<<<<<<<<<<<<<<<<<<<<

.include "m2560def.inc"
ldi r16, 0xFF 		;Setting up the data direction for Port B
out DDRB, r16 		;Set port B as output

ldi r16, 0x00 		;Setting up the data direction for Port A
out DDRA, r16 		;Set Port A as output

my_loop:		;Loop to always check which switch is pressed
	in r17,PINA ;Getting the information of which switch is pressed
	out portB, R17 	;Lighting up the corresponding LED
rjmp my_loop
\end{lstlisting}
\newpage
This is the flowchart for Task 2:
\begin{center}
\includegraphics{img/Task2.png}
\end{center}
\newpage
\section{Task 3}
\textit{Write a program in Assembly language to read the switches and light LED0 when you
press SW5.
For all other switches there should be no activity.}

\lstset{style=Asm}
\begin{lstlisting}
;>>>>>>>>>>>>>>>>>>>>>>>>>>>>>>>>>>>>>>>>>>>>>>>>>>>>>>>>>>>
; 1DT301, Computer Technology I
; Date: 2019-09-09
; Author:
; Loic GALLAND
; Leonardo PEDRO
;
; Lab number: 1
; Title: How to use the PORTs. Digital input/output. Subroutine call.
;
; Hardware: STK600, CPU ATmega2560
;
; Function: Program to only light up LED number 0 if the switch number 5 is pressed. If any other switch is pressed, nothing will happen

; Input ports: The Port A will be used as an input port in this Task
;
; Output ports: The portB is used as an output port
;
; Subroutines: If applicable.
; Included files: m2560def.inc
;
; Other information:
;
; Changes in program: (Description and date)
;
ldi r16, 0xFF 		;Setting up the data direction for Port B
out DDRB, r16 		;Set port B as output

ldi r16, 0x00 		;Setting up the data direction for Port A
out DDRA, r16 		;Set Port A as output

ldi r16, 0xFF		;Turn off all the LEDs
out portB, r16

ldi r18, 0b11011111	;Desired binary code for SWITCH number 5
ldi r19, 0b11111110	;Binary code to light up LED0

my_loop:
	in r17, PINA 	;get the info from the switch
	cp r17, r18 	;compare switch info with desired one 
	breq light 	;condition if r17=r18 go to the "light"
rjmp my_loop

light: out portB,r19	;turns on the LED0
\end{lstlisting}
\newpage
This is the flowchart for Task 3:
\begin{center}
\includegraphics[scale=0.7]{img/Task3.png}
\end{center}

\newpage
\section{Task 5}
\textit{Write a program in Assembly language that creates a Ring Counter. The values should be
displayed with the LEDs. Use shift instructions, LSL or LSR.
Make a delay of approximately 0.5 sec in between each count. Write the delay as a
subroutine.}
\lstset{style=Asm}
\begin{lstlisting}
;>>>>>>>>>>>>>>>>>>>>>>>>>>>>>>>>>>>>>>>>>>>>>>>>>>>>>>>>>>>
; 1DT301, Computer Technology I
; Date: 2019-09-09
; Author:
; Loic GALLAND
; Leonardo PEDRO
;
; Lab number: 1,  How to use the PORTs. Digital input/output. Subroutine call.
; Hardware: STK600, CPU ATmega2560
;
; Function: Create a program that creates a Ring Counter with a delay of 0.5 seconds. 
;
; Output ports: The portB is used as an output port
;
; Subroutines: A subroutine will be used when creating the delay. 
;<<<<<<<<<<<<<<<<<<<<<<<<<<<<<<<<<<<<<<<<<<<<<<<<<<<<<<<<<<<
.includes "m2560def.inc"
;Initialize SP, Stack Pointer
ldi r20, HIGH(RAMEND)	; R20 = high part of RAMEND address
out SPH,R20 		; SPH = high part of RAMEND address
ldi R20, low(RAMEND) 	; R20 = low part of RAMEND address
out SPL,R20 		; SPL = low part of RAMEND address

ldi r16, 0xFF 		;Setting up the data direction for Port B
out DDRB, r16 		;Set port B as output
ldi r17, 0b11111111 	;Initial LED state
out PortB, r17

ldi r22, 0xFF

my_loop:
	out portB, r17	;Output the current binary to PortB
	rol r17	;rotate the 0 to make it do the ring counter
	CALL Delay	;Delay of 0.5 second
	cp r17, r22	;compare r17 and r22.
	breq reset	;if equal go to reset method
rjmp my_loop

reset:
	rol r17		;rol one more time to remove the state with all lights off
	out portB, r17
	rjmp my_loop

Delay:
;Generated by delay loop calculator
:at http://www.bretmulvey.com/avrdelay.html
;Delay 4 050 000 cycles
;500ms at 8.1 MHz
	ldi r18, 21
	ldi r19, 140
	ldi 20, 174
L1:	dec r20
	brne L1
	dec r19
	brne L1
	dec r18
	brne L1
	rjmp PC+1
RET
\end{lstlisting}
\newpage
This is the flowchart of the task 5:
\begin{center}
\includegraphics[scale=0.7]{img/Task5.png}
\end{center}
\newpage
\section{Task 6}
\textit{Write a program in Assembly language that creates a Johnson Counter in an infinite loop.}

\lstset{style=Asm}
\begin{lstlisting}
;>>>>>>>>>>>>>>>>>>>>>>>>>>>>>>>>>>>>>>>>>>>>>>>>>>>>>>>>>>>
; 1DT301, Computer Technology I
; Date: 2019-09-09
; Author:
; Loic GALLAND
; Leonardo PEDRO
;
; Lab number: 1
; Title: How to use the PORTs. Digital input/output. Subroutine call.
;
; Hardware: STK600, CPU ATmega2560
;
; Function: Creates a program that creates a Johnson Counter in an infinite loop

; Input ports: NO inputs ports in this Task
;
; Output ports: The portB is used as an output port
;
; Subroutines: To be able to use the delay
; Included files: m2560def.inc
;<<<<<<<<<<<<<<<<<<<<<<<<<<<<<<<<<<<<<<<<<<<<<<<<<<<<<<<<<<<

.includes "m2560def.inc"

; Initialize SP, Stack Pointer
ldi r20, HIGH(RAMEND)	; R20 = high part of RAMEND address
out SPH,R20 		; SPH = high part of RAMEND address
ldi R20, low(RAMEND) 	; R20 = low part of RAMEND address
out SPL,R20 		; SPL = low part of RAMEND address

ldi r16, 0xFF	;Setting up the date direction register for Port B
out DDRB, r16	;Set port B as output

ldi r16, 0xFF	;Turn off all the lights
out portB, r16	;

ldi r21, 0b11111110	;Initial LED state
ldi r22, 0xFF	;When all the LEDs are turned off 
ldi r23, 0x00	;When all the LEDs are turned on

my_loop:
	out portB, r21	;Output r21 to portB
	LSL r21		;Shift all the bits to the left
	CALL Delay		;Delay of 0.5 seconds
	;Compare the current status of the LEDs to check if they are all turned on. 
	cp r21, r23	
	breq light	;When r21=R23 go to method "light"
rjmp my_loop




light:
	out portB, r23	;output r23 to portB
	CALL Delay		;Delay of 0.5s
	ldi r21, 0b10000000	;initialize the LED to make it go right
	out portB, r21
	Second_loop:
		out portB, r21	;output r21 to portB
		ASR r21	;Shift all the bits to the right so here it would go from 0b1000 0000 to 0b1100 000 and so on
		CALL Delay ;we call the delay
		cp r21, r22		;Compare the current status to know if it needs to start going the other way
				;So for the first one we compare 0b1100 0000 with 0b1111 1111
		breq my_loop ; if r21 == r22 which means all lights are turned off we go back to my_loop
	rjmp Second_loop ;if breq is not true then we repeat the loop of Second_loop

Delay:
;Generated by delay loop calculator
:at http://www.bretmulvey.com/avrdelay.html
;
;Delay 4 050 000 cycles
;500ms at 8.1 MHz

	ldi r18, 21
	ldi r19, 140
	ldi 20, 174
L1:	dec r20
	brne L1
	dec r19
	brne L1
	dec r18
	brne L1
	rjmp PC+1
RET
\end{lstlisting}
\newpage
This is the flowchart of the task 6:
\begin{center}
\includegraphics[scale=0.7]{img/Task6.png} 
\end{center}

% Prints your bibliography database xxx.bib
\bibliographystyle{IEEEtran}
\bibliography{ref.bib}

\end{document}
