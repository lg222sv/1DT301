%----------------------------------------------------------------------------------------
%
% A LaTeX-template for 1DV510. Modified and translated by Björn Lindenberg at LNU.
% Based on an original master thesis template created by Marcus Wilhelmsson at LNU.
%
%----------------------------------------------------------------------------------------

% Settings and document configuration

\documentclass[a4paper,12pt]{article} 
\usepackage[T1]{fontenc} 
\usepackage{times} 
\usepackage[swedish,english]{babel} 
\usepackage[utf8]{inputenc} 
\usepackage{dtk-logos} 
\usepackage{wallpaper} 
\usepackage[absolute]{textpos} 
\usepackage[top=2cm, bottom=2.5cm, left=3cm, right=3cm]{geometry} 
\usepackage[parfill]{parskip} 
\usepackage{csquotes} 
\usepackage{float} 
\usepackage{lipsum} % Used for dummy text. Can be removed.
\usepackage{listings, color}

\lstdefinestyle{Asm}{
  belowcaptionskip=1\baselineskip,
  breaklines=true,
  frame=L,
  xleftmargin=\parindent,
  language=[x86masm]Assembler,
  showstringspaces=false,
  basicstyle=\footnotesize\ttfamily,
  keywordstyle=\bfseries\color{purple!40!black},
  commentstyle=\itshape\color{green!40!black},
  identifierstyle=\color{blue},
  stringstyle=\color{orange},
}

% Fontsizes for section headings.
\usepackage{sectsty} 
\sectionfont{\fontsize{14}{15}\selectfont}
\subsectionfont{\fontsize{12}{15}\selectfont}
\subsubsectionfont{\fontsize{12}{15}\selectfont}

%----------------------------------------------------------------------------------------
%	This part is used for the text box on the title page
%----------------------------------------------------------------------------------------
\newsavebox{\mybox}
\newlength{\mydepth}
\newlength{\myheight}

\newenvironment{sidebar}%
{\begin{lrbox}{\mybox}\begin{minipage}{\textwidth}}%
{\end{minipage}\end{lrbox}%
 \settodepth{\mydepth}{\usebox{\mybox}}%
 \settoheight{\myheight}{\usebox{\mybox}}%
 \addtolength{\myheight}{\mydepth}%
 \noindent\makebox[0pt]{\hspace{-20pt}\rule[-\mydepth]{1pt}{\myheight}}%
 \usebox{\mybox}}

%----------------------------------------------------------------------------------------
%	Title
%----------------------------------------------------------------------------------------
\newcommand\BackgroundPic{
    \put(-2,-3){
    \includegraphics[keepaspectratio,scale=0.3]{img/lnu_etch.png} % Background image
    }
}
\newcommand\BackgroundPicLogo{
    \put(30,740){
    \includegraphics[keepaspectratio,scale=0.10]{img/logo.png} % LNU logo
    }
}

\title{
\vspace{-8cm}
\begin{sidebar}
    \vspace{10cm}
    \normalfont \normalsize
    \huge Computer Technology I\\ % Main title
    \vspace{-1.3cm}
\end{sidebar}
\vspace{3cm}
\begin{flushleft}
    \huge Lab. 2 : Subroutines % Subtitle
     \small \\ \emph{}
\end{flushleft}
\null
\vfill
\begin{textblock}{5}(10,13)
\begin{flushright}
\begin{minipage}{\textwidth}
\begin{flushleft} \large
\emph{Author:}\textsc{ Loic GALLAND, Leonardo PEDRO}\\  % Author
\emph{Supervisor:}  \textsc{} \\  % Author
\emph{Semester:} Autumn 2019\\ % Semester
\emph{Area:} Computer Science \\ % Area
\emph{Course code:} 1DT301 % Course
\end{flushleft}
\end{minipage}
\end{flushright}
\end{textblock}
}

\date{} % Empty date command. Use \today inside for today's date.
\author{} % Normally one would use this to define authors. However in this case the title command takes care of everything, so we leave the field empty to get rid of warnings. 

\begin{document}

\pagenumbering{gobble} % Turn off page numbering
\newgeometry{left=5cm}
\AddToShipoutPicture*{\BackgroundPic} % Adds the background image to the title page
\AddToShipoutPicture*{\BackgroundPicLogo} % Adds the logo to the title page
\maketitle % Prints the title
\restoregeometry
\clearpage

\pagenumbering{roman} % Roman page numbering for abstract page


\selectlanguage{english}

\newpage

\pagenumbering{gobble} % Turn off page numbering
\tableofcontents 

\newpage
\pagenumbering{arabic} % Turn on page numbering


\section{Task 1 - Switch – Ring counter / Johnson counter}

\textit{Write a program which switch between Ring counter and Johnson counter. You should not use
Interrupt in this lab. The pushbutton must be checked frequently, so there is no delay between the
button is pressed and the change between Ring/Johnson. Use SW0 (PA0) for the button. Each
time you press the button, the program should change counter.}

\lstset{style=Asm}
\begin{lstlisting}
WRITE SHIT HERE
\end{lstlisting}
This is the flowchart of the task 1:
\begin{center}
%\includegraphics[scale=0.7]{img/Task1.png}
\end{center}
\newpage
\section{Task 2 - Electronic dice}
\textit{You should create an electronic dice. Think of the LEDs placed as in the picture below. The
number 1 to 6 should be generated randomly. You could use the fact that the time you press the
button varies in length.}

\lstset{style=Asm}
\begin{lstlisting}

\end{lstlisting}

This is the flowchart of the task 2:
\newpage
\begin{center}
%\includegraphics[scale=0.4]{img/Task2.png}
\end{center}

\newpage
\section{Task 3 - Change counter}
\textit{Write a program that is able to count the number of changes on a switch. As a change we count
when the switch SW0 goes from 0 to 1 and from 1 to 0, we expect therefore positive and negative
edges. We calculate the changes in a byte variable and display its value on PORTB.}

\lstset{style=Asm}
\begin{lstlisting}

\end{lstlisting}

\newpage
This is the flowchart of the task 3:
\begin{center}
%\includegraphics[scale=0.8]{img/TASK3.png}
\end{center}

\newpage
\section{Task 4 - Delay subroutine with variable delay time}

\lstset{style=Asm}
\begin{lstlisting}

\end{lstlisting}

\newpage
This is the flowchart of the task 4:
\begin{center}
%\includegraphics[scale=0.7]{img/Task4.png}
\end{center}
% Prints your bibliography database xxx.bib
\bibliographystyle{IEEEtran}
\bibliography{ref.bib}

\end{document}
